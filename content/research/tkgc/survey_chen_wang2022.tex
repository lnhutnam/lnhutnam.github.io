\documentclass[12pt]{article}

\usepackage{fullpage}
\usepackage{amsmath,amssymb,amsfonts,amsthm}

\usepackage{ebgaramond}
\usepackage[cmintegrals,cmbraces]{newtxmath}
\usepackage{ebgaramond-maths}

\usepackage{braket}
\DeclareMathOperator{\cas}{cas}
\newcommand{\trans}[1]{\ensuremath{{#1}^\intercal}}

\usepackage{color}

\setlength\parskip{2mm}

\usepackage{hyperref}
\usepackage[style=authoryear,backend=bibtex]{biblatex} %backend tells biblatex what you will be using to process the bibliography file
\addbibresource{refs.bib}

\begin{document}
\title{A Survey on Temporal Knowledge Graphs-Extrapolation and Interpolation Tasks}
\author{Sulin Chen, and Jingbin Wang}
\maketitle

\section{What?}

This paper provided a comprehensive study of temporal knowledge graph, discussing the main existing temporal models and techniques
which divided into extrapolation tasks and interpolation tasks. 

\section{Why?}

The traditional KGE methods do not consider the time factor. And in the real world, data always involves larger and larger 
through the time. These old method get hard to expand on the new scenarios.

\section{How?}

TKG: $\{(e_i, r, e_j, t)\}$

In context of learning and KGC, entities and relationships of the KG undergo dynamic modifications over time.


\subsection{Backgrounds}

Temporal point processes can be represented based on the conditional strength function $\lambda(t)$ as stochastic model of the 
next event time given all previous events,

\begin{equation}
    \lambda(t)dt := \mathbf{P}\{eventin[t, t + dt] \mid T(t)\} = \mathbf{E}[dN(t) \mid T(t)]
\end{equation}
where:
\begin{itemize}
    \item $\lambda(t)dt$ the conditional probability of observing an event in a small window given the history up to $t$.
    \item A small window $ = [t, t + dt]$.
    \item The history up to $t$, i.e, $T(t) := \{t_{\tau} < t\}$.
    \item $dt$ is a small window of size which only one event can occur, i.e, $dN(t) \in \{0, 1\}$.
\end{itemize}

The conditional density of the event occurring at moment $t$ is defined as:

\begin{equation}
    h(t) = \lambda(t)S(t)
\end{equation}
where:
\begin{itemize}
    \item $S(t)$ is the conditional probability that no event occurs.
\end{itemize}

\subsection{Extrapolation .vs Interpolation}

Extrapolation and interpolation are both used to estimate hypothetical values for a variable based on other observations.
\begin{itemize}
    \item In interpolation setting, we could use our model to predict/ estimate the value of the dependent variable for an 
    independent variable that in our data.
    \item In extrapolation setting, we could use our model to predict/ estimate the value of the dependent variable for an 
    independent variable that is outside the range of our data.
\end{itemize}

\textbf{Note that}: In extrapolation, we are making the assumption that our observed trend continues for values in the outside 
the range we used to form our model.

\subsection{Temporal knowledge graphs - Extrapolation setting}

\subsubsection{Temporal point process}

The combination of temporal point process and deep networks to procedure neural point process which demonstrated 
powerful capabilities.
\begin{itemize}
    \item Know-Evolve (\cite{trivedi2017know})
    \item DeRep
\end{itemize}

\subsubsection{Time series models}

\begin{itemize}
    \item RE-Net
    \item CyGNet
    \item HIPNET
    \item CluSTeR
    \item DySAT
    \item TemporalGAT
    \item FTAG
    \item xERTE
\end{itemize}

\subsubsection{Others}

\begin{itemize}
    \item TGAT
\end{itemize}

\subsection{Temporal knowledge graphs - Interpolation setting}

\subsubsection{Translational distance model}

\begin{itemize}
    \item t-TransE, TTransE are based on TransE model.
    \item HyTE is based on TransH model.
\end{itemize}

\subsubsection{Semantic matching model}

\begin{itemize}
    \item TComplEx
    \item TIMEPLEX
    \item TeLM
\end{itemize}

\subsubsection{Neural network model}

\begin{itemize}
    \item TeMP
\end{itemize}

\subsubsection{Relational rotation model}

\begin{itemize}
    \item TeRo
    \item ChronoR
\end{itemize}

\subsubsection{Hyperbolic geometric model}

\begin{itemize}
    \item DyERNIE (\cite{han2020dyernie})
    \item HERCULES ()
\end{itemize}

\nocite{*}
\printbibliography

\end{document}