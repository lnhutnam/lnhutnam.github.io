\documentclass[12pt]{article}

\usepackage{fullpage}
\usepackage{amsmath,amssymb,amsfonts,amsthm}

\usepackage{ebgaramond}
\usepackage[cmintegrals,cmbraces]{newtxmath}
\usepackage{ebgaramond-maths}
\usepackage[backend=bibtex]{biblatex}

\usepackage{braket}
\DeclareMathOperator{\cas}{cas}
\newcommand{\trans}[1]{\ensuremath{{#1}^\intercal}}

\usepackage{color}

\setlength\parskip{2mm}

\begin{document}
\title{Involutory Property of the Discrete Hartley Transform}
\author{Frank the Giant Bunny}
\maketitle

Given a column vector $x\in\mathbb{R}^n$, its \emph{Discrete Hartley Transform}
(DHT) is defined as another vector $y\in\mathbb{R}^n$ such that
\begin{equation}
  y_j = \frac{1}{\sqrt n}\sum_{i=0}^{n-1}x_i
  \cas\left(\tfrac{2\pi}{n}ij\right)
  \quad\text{for}\quad j\in\set{0,\cdots,n-1}
  \label{eq:DHT}
\end{equation}
where the $\cas$ function is defined as $\cas\theta = \cos\theta + \sin\theta$.
Interestingly, the DHT is an \emph{involution}; that is, the DHT is the same as the
inverse DHT.
\begin{equation}
  x_i = \frac{1}{\sqrt n}\sum_{j=0}^{n-1}y_j
  \cas\left(\tfrac{2\pi}{n}ij\right)
  \quad\text{for}\quad i\in\set{0,\cdots,n-1}
  \label{eq:IDHT}
\end{equation}
This paper proves the DHT is indeed an involution.

\noindent\textbf{\textsf{Target Equality}}\quad
To simplify \eqref{eq:DHT} and \eqref{eq:IDHT},
define an $n\times n$ symmetric matrix $H$ whose $(i,j)$-entry is
$\frac{1}{\sqrt n}\cas\left(\frac{2\pi}{n}ij\right)$.
Then the DHT and the inverse DHT are 
\begin{equation*}
  y = Hx
  \quad\text{and}\quad
  x = Hy
\end{equation*}
where $x=\trans{
  \begin{bmatrix}x_0,\cdots,x_{n-1}\end{bmatrix}
}$ and
$y=\trans{
  \begin{bmatrix}y_0,\cdots,y_{n-1}\end{bmatrix}
}$.
Then proving the involutory property of the DHT reduces to showing that 
$H^2 = I$ where $I$ is an $n\times n$ identity matrix.
This further reduces to showing that the rows in $H$ are orthogonal;
that is,
\begin{equation*}
  \braket{h_i, h_{i'}} = 
  \begin{cases}
    1, &\text{if $i=i'$;} \\
    0, &\text{otherwise;}
  \end{cases}
\end{equation*}
where $h_i=\tfrac{1}{\sqrt n}\trans{
  \begin{bmatrix}
    \cas\left(\tfrac{2\pi}{n}i0\right),
    \cdots,
    \cas\left(\tfrac{2\pi}{n}i(n-1)\right)
  \end{bmatrix}}$
is the $i^{\mathrm{th}}$ row in $H$.
It may be written in terms of $\cas$ functions as
\begin{equation}
  \sum_{j=0}^{n-1}
  \cas\left(\tfrac{2\pi}{n}ij\right) \cas\left(\tfrac{2\pi}{n}i'j\right)
  =
  \begin{cases}
    n, &\text{if $i=i'$;} \\
    0, &\text{otherwise;}
  \end{cases}
  \label{eq:target}
\end{equation}
which is the target equality for the involutory property.

\noindent\textbf{\textsf{CAS Identity}}\quad
The proof of \eqref{eq:target} begins with an identity about $\cas$ functions.
\begin{equation}
  \cas\alpha\cas\beta = \sin(\alpha+\beta) + \cos(\alpha-\beta)
  \label{eq:tool}
\end{equation}
\begin{proof}[Proof of \eqref{eq:tool}]
A $\cas(\cdot)$ can be simplified to $\cos(\cdot)$ as follows:
\begin{equation}
  \cas\theta
  =
  \cos\theta + \sin\theta
  =
  \sqrt{2}\left(
    \frac{1}{\sqrt 2}\cos\theta + \frac{1}{\sqrt 2}\sin\theta
  \right)
  =
  \sqrt{2}\cos\left(\theta - \frac{\pi}{4}\right)
  \tag{$\ast$}
  \label{eq:cas}
\end{equation}
Then
\begin{alignat*}{2}
  \cas\alpha\cas\beta
  &=
  \sqrt{2}\cos\left(\alpha - \frac{\pi}{4}\right)
  \cdot
  \sqrt{2}\cos\left(\beta - \frac{\pi}{4}\right)
  &
  \quad\quad\text{by \eqref{eq:cas}}
  \\
  &=
  \cos\left(\alpha + \beta - {\frac \pi 2}\right)
  +
  \cos(\alpha - \beta)
  &
  \quad\quad
  2\cos\theta\cos\phi
  =
  \cos(\theta+\phi) + \cos(\theta-\phi)
  \\
  &=
  \sin(\alpha + \beta) + \cos(\alpha - \beta)
  &
  \quad\quad
  \cos\left(\theta-\frac{\pi}{2}\right) = \sin\theta
\end{alignat*}
which completes the proof.
\end{proof}

\noindent\textbf{\textsf{Target Simplified}}\quad
The target equality \eqref{eq:target} is decomposed into two summations by
\eqref{eq:tool}.
\begin{equation*}
  \sum_{j=0}^{n-1}
  \cas\left(\tfrac{2\pi}{n}ij\right) \cas\left(\tfrac{2\pi}{n}i'j\right)
  =
  \sum_{j=0}^{n-1} \sin\left(\tfrac{2\pi}{n}(i+i')j\right)
  +
  \sum_{j=0}^{n-1} \cos\left(\tfrac{2\pi}{n}(i-i')j\right)
\end{equation*}
The above will be interpreted as the real and imaginary parts in 
geometric progressions of complex numbers:
\begin{equation}
  \sum_{j=0}^{n-1}
  \cas\left(\tfrac{2\pi}{n}ij\right) \cas\left(\tfrac{2\pi}{n}i'j\right)
  =
  \Im\left\{\sum_{j=0}^{n-1} \omega^{(i+i')j} \right\}
  +
  \Re\left\{\sum_{j=0}^{n-1} \omega^{(i-i')j} \right\}
  \label{eq:geom}
\end{equation}
where $\omega$ is the \emph{primitive $n^{\mathrm{th}}$ root of unity}
$\omega=\exp(\iota 2\pi/n) = \cos(2\pi/n) + \iota\sin(2\pi/n)$.
The identity \eqref{eq:geom} is easily proved by the De Moivre's identity.

\noindent\textbf{\textsf{Summation Lemma}}\quad
The last puzzle to the involutory property proof is the summation
of a geometric series:
\begin{equation*}
  \text{For an integer $k$,}\quad
  \sum_{j=0}^{n-1}(\omega^k)^j = 
  \begin{cases}
    n, & \text{if $k$ is a multiple of $n$;} \\
    0, & \text{otherwise.}
  \end{cases}
\end{equation*}
This lemma ensures that the imaginary part of $\sum_{j=0}^{n-1}(\omega^k)^j$ is
zero regardless of the integer value $k$. On the other hand,
the real part is $n$ if $k$ is a multiple of $n$ and zero otherwise.
Plugging in theses values to the RHS of \eqref{eq:geom} yields the desired
identity \eqref{eq:target}, which completes the involutory property proof.
\end{document}
